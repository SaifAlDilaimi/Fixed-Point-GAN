\chapter{Results}
\label{chapter_results}
The main aim of this work was to increase the amount of disease entities by replacing the \ac{UniProt} database. To find a suitable database, a research must be carried out to find databases that meet the requirements. As already mentioned \ac{BIONDA} is using the database \ac{UniProt}, which does not contain a large amount of disease entries. In addition it does not have any or very few synonyms, which is a problem as they are not linked to each other and therefore it is harder for scientists to recognize relations between diseases. The alternative database needs to fulfill those two main features. This research process has identified a few candidates that fulfill these requirements. However, all candidates and other requirements will be presented in the following section.

\section{Overview of Available Disease Databases}
The first candidate is Disease Ontology. This database contains 11,924 disease entries, which is more than double the number of \ac{UniProt}. It is provided by the Institute of Genome Sciences University of Maryland School of Medicine and is updated frequently \citep{noauthor_disease_nodate}. Similar, to the \ac{UniProt} database it is also freely accessible. A second candidate is the \ac{MeSH} database, which is also freely accessible. The exact number of disease entries is unfortunately not apparent, since different fields of medicine are integrated in this database. These include types of injuries, proteins and also other areas that are irrelevant for this project. This makes it necessary to extract the diseases from any other field, which complicates the automation of this process. Identical to the Disease Ontology, this database is provided by a public organization, too, namely the U.S. Department of Health \& Human Services \citep{noauthor_mesh_nodate}. \\

The database Malacards has about 22,000 disease entries, which has the most entries of all considered database. It is also provided by a public organization, the Weizmann Institute of Science. However, it is not freely accessible, which is essential for the integration into \ac{BIONDA} \citep{noauthor_malacards_nodate}. One other database is Open Targets. This database is freely accessible as well and offers 12,422 disease entries, which is again more than twice of \ac{UniProt}. Open Targets is provided by GSK, EMBL-EBI, Sanger, Snnofi, Biogen, Takeda and Celgene \citep{noauthor_opentargets_nodate}. All of these companies are big players in the biomedical scene. Based on these properties the following databases will be taken into consideration. 

\begin{itemize} 
\item \ac{UniProt} 
\item Open Targets 
\item \ac{MeSH} 
\item Disease Ontology 
\end{itemize}

All of these databases could be a suitable replacement of \ac{UniProt}. The main target of getting more disease entries will be fulfilled with each of these. 

\section{Comparison of Available Disease Databases}
Based on these criteria and their weights these databases can now be evaluated. It is immediately noticeable that Malacards is not freely accessible and therefore can be excluded from the comparison of the given databases. The other databases fulfill most of the established criteria and can now be tested.

\begin{table}[H]
\centering
\resizebox{\textwidth}{!}{%
\begin{tabular}{|l|c|c|c|c|c|}
\hline
\multirow{2}{*}{\textbf{Criteria}} & \multicolumn{1}{l|}{\multirow{2}{*}{\textbf{Weight}}} & \multicolumn{4}{c|}{\textbf{Databases}}                                                                                                                                  \\ \cline{3-6} 
                                   & \multicolumn{1}{l|}{}                                 & \multicolumn{1}{l|}{\textbf{Open Targets}} & \multicolumn{1}{l|}{\textbf{Disease Ontology}} & \multicolumn{1}{l|}{\textbf{MeSH}} & \multicolumn{1}{l|}{\textbf{UniProt}} \\ \hline
Number of entries                  & x4                                                    & 8                                          & 10                                             & 4                                  & 5                                     \\ \hline
Synonyms                           & x5                                                    & 2                                          & 10                                             & 4                                  & 8                                     \\ \hline
Freely accessible                  & x6                                                    & 10                                         & 10                                             & 10                                 & 10                                    \\ \hline
Public organization                & x6                                                    & 10                                         & 10                                             & 10                                 & 10                                    \\ \hline
API (unlimited)               & x3                                                    & 5                                          & 0                                              & 3                                  & 5                                     \\ \hline
Update frequency                   & x4                                                    & 6                                          & 10                                             & 9                                  & 6                                     \\ \hline
\multicolumn{2}{|l|}{\textbf{Points scored}}                                               & \textbf{201/270}                           & \textbf{240/270}                               & \textbf{201/270}                   & \textbf{219/270}                      \\ \hline
\end{tabular}%
}
\caption{Evaluation Matrix of Compared Databases}
\label{tab:evaluationmatrix}
\end{table}

However, it is important to check raw data of these databases in order to test how significant the properties are. After a check of the given data it is clear that Disease Ontology (see appendix \ref{Anhang_PatterMatching}) gives the best result. The Disease Ontology is a standardized community driven database for biomedical data dealing with human diseases. Its aim is to provide consistent, reusable and sustainable descriptions of human disease terms, phenotype characteristics and related medical vocabulary disease concepts for clinical and medical research \citep{do}. Disease Ontology is updated frequently through cross-referencing of terms to other databases like \ac{MeSH}, \ac{OMIM}, \ac{GARD}, Orphanet and a lot of other vocabularies. This cross-referencing allows frequent updates of diseases and other medical terms. However, all releases can be found on GitHub, too. For each disease the database provides an DOID, a Disease Ontology unique ID, a disease term name, whether or not the disease is obsolete and a definition. Further it provides Xrefs which are cross-references from other clinical vocabularies, alternative ids which are merged DOIDs, synonyms of each disease and a parent-child-relationship, which connects diseases with each other. The amount of diseases is more than twice  of the \ac{UniProt} database. Furthermore, the relationships between diseases are given in Disease Ontology, which would help to visualize their relations, e.g. as a tree. All of these attributes are very important for the database later on, i.e. could improve and expand the functions of \ac{BIONDA} in a future release. \\

\section{Comparison UniProt vs. Disease Ontology}
However, one needs to discuss how effective the replacement is. As a matter of fact due to the new source database the available disease entities have increased by $113.73\%$ from 5,466 to 11,683. As already mentioned the fact that Disease Ontology contains synonyms was one of the key features to consider it as a suitable disease database. At the beginning of this project the \ac{UniProt} database did not contain any synonyms, making the decision to choose Disease Ontology unworthy to discuss. However, during the staging phase of this project \ac{UniProt} released a new version of its human disease database. This fact makes it necessary to compare random diseases and their properties from both databases with each other. Especially one needs to compare how naively their synonyms are. This is truly significant as BIONDA, whose target groups are not only scientists but also people without any scientific background. Table \ref{tab:ComparisionUniProtDiseaseOntology} compares three diseases each from the \ac{UniProt} and Disease Ontology database. One can observe that the Disease Ontology database contains more synonyms on each disease compared to the \ac{UniProt} database. They also seem to sound more intuitively, which makes it possible for non scientists to get search results.  An important point to mention is that the raw data of the database contains multiple disease entries of the same disease. At first, those may seem as duplicates. However, on second sight, they resemble the same disease, yet in different stages. A good example for this case is Alzheimer, which is extremely dependent from which gene or chromosome it originates.\\

\begin{table}[H]
\centering
\resizebox{\textwidth}{!}{%
\begin{tabular}{|l|l|l|l|}
\hline
\multicolumn{1}{|c|}{\textbf{Property/Disease Database}} & \multicolumn{1}{c|}{\textbf{Disease 1}}                                                                                                                                                                                  & \multicolumn{1}{c|}{\textbf{Disease 2}}                                              & \multicolumn{1}{c|}{\textbf{Disease 3}} \\ \hline
\multicolumn{4}{|c|}{\textbf{UniProt}}                                                                                                                                                                                                                                                                                                                                                                               \\ \hline
Name                                                     & breast cancer                                                                                                                                                                                                            & Alzheimer                                                                            & malaria                                 \\ \hline
Number of Synonyms                                       & 4                                                                                                                                                                                                                        & 1                                                                                    & -                                       \\ \hline
Synonyms                                                 & \begin{tabular}[c]{@{}l@{}}"Breast cancer familial", \\ "Breast cancer \\ familial male", \\ "Breast carcinoma”, \\ "Mammary carcinoma"\end{tabular}                                                                     & \begin{tabular}[c]{@{}l@{}}"Presenile and \\ senile dementia"\end{tabular}           & -                                       \\ \hline
\multicolumn{4}{|c|}{\textbf{Disease Ontology}}                                                                                                                                                                                                                                                                                                                                                                      \\ \hline
Name                                                     & breast cancer                                                                                                                                                                                                            & Alzheimer                                                                            & malaria                                 \\ \hline
Number of Synonyms                                       & 6                                                                                                                                                                                                                        & 2                                                                                    & 1                                       \\ \hline
Synonyms                                                 & \begin{tabular}[c]{@{}l@{}}"breast tumor",\\ "malignant neoplasm\\ of breast",\\ "malignant tumor \\ of the breast",\\ "mammary cancer",\\ "mammary neoplasm",\\ "mammary tumor",\\ "primary breast cancer"\end{tabular} & \begin{tabular}[c]{@{}l@{}}"Alzheimer disease",\\ "Alzheimers dementia"\end{tabular} & "induced malaria"                       \\ \hline
\end{tabular}%
}
\caption{Comparision of synonyms for three diseases between UniProt and Disease Ontology}
Source: \citep{do, uniprot}
\label{tab:ComparisionUniProtDiseaseOntology}
\end{table}

However, these results are good indications that the decision to choose Disease Ontology was the right one. Nonetheless, it is still necessary to verify the success rate of this new data. To achieve this one needs to validate the new diseases with the help of the text mining component of BIONDA. This component is part of the infrastructure of BIONDA and cannot be accessed by the project team. Nonetheless, the project lead decided to take the extracted disease data and allow the text mining component to process them. The process will consider only abstracts of publications which were published in 2019. Including earlier years would increase the overall runtime without any statistical benefit. The returned results from the project lead did not include any disease synonyms. Even when excluded, the process did find 7,751 disease-biomarker pairs in abstracts, which is a $102.80\%$ increase. Another factor which gets evaluated is the unique number of diseases found through the process. \ac{UniProt} did find 5,986, which is not comparable to the 14,159 diseases found by Disease Ontology as a source database.\\

Overall, one can say that the replacement of the database is beneficial in all cases. Including all previous years will increase the search results even more. In the next chapter these results will be discussed.
