\chapter{Discussion}
The aims of this project were to compare the available source databases for diseases and to implement a mechanism which adds diseases to the existing BIONDA database. The first part of this work compared several databases based on specific requirements, which are of specific value to the BIONDA project team. The research and evaluation process of these databases was done in close cooperation with the project team. Especially, BIONDAs core values also needed to be represented by this new database. These discussions resulted in choosing the Disease Ontology database, which fulfilled all needed requirements (see figure \ref{tab:evaluationmatrix}) and matched BIONDAs core values. After this question is answered, the development of the update process was implemented in java as a background process. Many use cases were handled to make the process more robust. An example for that was to save the update progress and continue at the same state in case of an interruption. \\

This work was based on the premise to increase the amount of results by replacing the source database. This is successfully done by this project and one can observe how much more beneficial this change is, as detailed in chapter \ref{chapter_results}. The next step for this application will be the integration into the infrastructure of BIONDA. As mentioned, this process will take care of updating the BIONDA database with new diseases published by Disease Ontology. This process can be started as a daily cron job to populate the database. As part of this process the text mining component will parse all abstracts and build all scores for each new disease. This is an important step to determine how good the results over all years are. Now it is also possible to visualize the parent-child-relationships between diseases in BIONDA. Due to the representation of all disease relationships in the new database it is possible to track each relationship to the root or leaf disease. This, however, is not part of this work and will be part of the future development process of BIONDA. 