\chapter*{Produkteinsatz}

\section*{Zielgruppe}
Die Anwendung wird von Mitarbeitern des Forschungsbereichs \textit{Medizinische Bioinformatik}, ansässig am \textit{Medizinischen Proteom-Center},  verwendet. Zukünftige Zielgruppen sollen sowohl die Pharmaindustrie, Wissenschaftler weltweit als auch Betroffene sein.

\section*{Betriebsbedingung}
Die Anwendung ist in zwei Teilkomponenten aufgeteilt: Web-Applikation und Datenbank-Routine. Die Web-Applikation ist webbasiert und soll, bei noch vorhandenen zeitlichen Ressourcen, angepasst werden. Die Datenbank-Routine soll dahingehend angepasst werden, in dem eine alternative Krankheits-Datenbank als Quell-Datenbank integriert wird. Bedeutet eine Integration in die BIONDA-MySQL-Datenbank durch hinzufügen neuer Tabellen und Einträge in die Datenbank.\\

Folgende Programmiersprachen sollen hierbei zum Einsatz kommen:
\begin{itemize}
\item{Java}
\begin{itemize}
\item{Vorbereitung der Daten für die Datenbank, gegebenenfalls Perl, da zukünftig die Daten mit dieser Sprache vorbereitet werden sollen}
\end{itemize}
\item{PHP}
\item{Python}
\item{SQL}
\end{itemize}
