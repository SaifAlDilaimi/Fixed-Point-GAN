\chapter*{Qualitätsanforderung}
Im Folgenden werden die wesentlichen Qualitätsanforderungen erläutert.

\section*{Erweiterbarkeit}
Die zu entwickelnde Softwarestruktur muss so gewählt werden, dass eine einfache Implementierung der angedachten Erweiterungen möglich ist. 
Mögliche Erweiterungen oder Änderungen bereits implementierter Funktionen sollen mit geringem zeitlichem Aufwand durchgeführt werden können, ohne dass eine tief greifende Umstrukturierung des bestehenden Quellcodes vorgenommen werden muss. 

\section*{Fehlerrobustheit}
Alle während des Betriebes auftretende Fehler und Warnungen müssen abgefangen und zur Anzeige gebracht werden. Im Fehlerfall muss der laufende Prozess kontrolliert abgebrochen werden. Zusätzlich muss entschieden werden, ob die Schwere des Fehlers einen Programmabbruch zur Folge haben soll. Angefallene Fehler sollen in einer lokal gespeicherten Datei protokolliert werden. 

\section*{Wartbarkeit}
Um eine langfristige Nutzung der Software zu ermöglichen, muss das Projekt so aufgebaut und dokumentiert sein, dass die Administration, Wartung und Weiterentwicklung ohne großen Einarbeitungsaufwand von einem anderen Entwickler oder Administrator durchgeführt werden kann. 
Um dies zu gewährleisten, muss die Anwendung modular programmiert und exakt dokumentiert sein. 
Dazu gehören eine Projektdokumentation, eine Anwenderdokumentation und eine technische Dokumentation. 